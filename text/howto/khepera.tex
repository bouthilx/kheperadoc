\section{Khepera}
\label{sec:howto:khepera}

    \begin{enumerate}
        \item Find a color that is easy to detect given the colors of 
            the maze. Different textures or materials may have different 
            effectiveness. Drawing color on a white paper seems really 
            ineffective.
        \item If you use the OrientationHistogramDetector 
            (HistogramDetector and TriangleTracker), you will need 
            to get the Khepera out of the camera view every time you 
            start the detector. If you use the OrientationColorDetector 
            (ColorDetector and TriangleTracker), you can leave the robot 
            in the view of the detector when you start it, it is not a 
            problem.
        \item Put the Khepera on. 
        \item In Ubuntu, open the Bluetooth administration panel. If 
            Khepera does not appear in the detected devices click on 
            search. If it still does not appear, the battery may not be 
            full enough. Connect the Khepera on the charge and try 
            again. It should appear if it is on and plugged on the charge.
        \item Click on pair (key) if not paired. The access key is 0000
        \item Click on setup to setup the serial port.
        \item If it works, write down the port. (/dev/rfcomm0 most of the 
            time)
        \item If it does not work. Try the following : \\
            \url{http://www.k-team.com/forum/index.php?topic=553.0}
        \item To test the connection and to be able to stop the Khepera 
            if the \clsquare{} experiment fails, run 
                ./CLsquare manipulate\_robot.cls
        \item If the connection fails, try to use minicom as it is 
            described here:  \\
            \url{http://www.k-team.com/forum/index.php?topic=553.0}
        \item Once the connection is correctly established, try moving 
            the robot with commands D,l5000,l-5000. You can change 
            values 5000 and -5000 for any values you want between -20000 
            and 20000. 
        \item The Khepera is now ready.
    \end{enumerate}

For more information, take a look at the sections \ref{sec:ohd}, 
\ref{sec:ocd} and \ref{sec:dk}.
