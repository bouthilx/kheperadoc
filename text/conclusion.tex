\chapter{Conclusion}
\label{sec:conclusion}

I have explained different modules for detection. There was first two 
detectors for the orientation, from which the second one, the color 
detector, proved to be more stable. I then explained how the colored 
grid detector and the maze detector work in order to detect a maze made 
of black tape on a white board.

To use the detectors, I needed to build new classes for the Maze-Khepera 
plant. From the major ones, a class, the MazeKheperaData, wraps the 
maze data structure to enable reading from detector rather than static 
file. But the most important one is the DirectedKhepera class that 
enables to move the Khepera to specific ($x_i$, $y_i$) positions by 
communicating with it via Bluetooth. 
\\
\\
The system works pretty well with no major flaws. The robot sometimes 
gets out of the maze but it is quite unusual. Even though, there are 
still lots of improvements that could be done, as the detectors sometimes 
get unstable and the robot navigates really slowly. 
\\
\\
The project could be pushed farther. We could add the possibility to 
work with varying target, varying in intensity for a weighted reward or 
varying in positions. Also, we could enhance the plant and make it 
continuous. The Khepera would then have more flexibility for its 
movements and they may look more natural. There are plenty of 
possibilities with this setup. I hope the structure I made will be 
useful to build up new ones on it. 
