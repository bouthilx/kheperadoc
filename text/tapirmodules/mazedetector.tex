\section{Maze Detector}

\subsection{Introduction}

The colored grid detector worked well to detect mean colors of the blocks, 
but it was too unstable in some circumstances, around the target for 
example. The mean color sometimes alternates quickly between two or 
three mean colors. To fix this, I built a subdetector that has two 
method of detection. First it detect the mean color and the closest 
mean color as for the Colored grid detector, but if the closest mean 
color is not close enough (distance bigger than max\_distance\_to\_color) 
then the detected color is the closest one between the wall color, by 
default black, and the ground color, by default white.   

\subsection{--------- (Algorithm)}

    \begin{enumerate}
        \item The detector enhance the colors given the \_saturation and 
            \_brightness values.
        \item It calculates the mean color of every block
        \item It draw the mean color as a thick border inside blocks of 
            the raw image
        \item It calculates the closest color from the mean block color
        \item Set block color
        \begin{enumerate}
            \item If the closest color is closest than a given threshold, 
                block color is set to the closest color value
            \item If the closest color is more far than the given 
                threshold, block color is set to closest color between 
                wall color and ground color
        \end{enumerate}
        \item It draws the block color in the edited image (under the 
            raw image)
        \item Add blocks to the detected objects
        \begin{enumerate}
            \item x,y (upper left corner)
            \item orientation=0, size=color index (ground=0,wall=1)
        \end{enumerate}
        \item Draw the grid on the raw an edited image
        \item If in get\_block\_color mode, it draws the mean color of 
            the (get\_block\_x,get\_block\_y) block in the edited image.
    \end{enumerate}

\subsection{Improvements}

The same improvements of the colored grid detector apply to this one.

\subsection{How to use}
    \begin{enumerate}
        \item Find colors that are easy to detect. Typically black and 
            white for the walls and the free space, and a basic color 
            like red for the target. An easy way to build it is with 
            red and black tape on a white board.
        \item You can build the maze before, but it is easier too build 
            it while the detector is on, so you can keep track of how 
            well it is detected.
        \item Once the camera is placed above the maze or the empty board, 
            set the xmin,xmax,ymin,ymax values using the arrows to narrow 
            down the subframe around the maze. Be sure that the terminal 
            as the focus. Print on screen the value, write it down and 
            save them in the configuration file.
        \item Use get\_block\_color mode to get the color values (RGB) 
            of the wall, the ground and the target. Wall and ground 
            colors are black and white by default.
        \item You can adjust the brightness and saturation values to 
            get better results
    \end{enumerate}

    \subsubsection{Parameters config file}
        \begin{description}
            \item[grid\_x] \hfill \\ int, Number of blocks in x
            \item[grid\_y] \hfill \\ int,  Number of blocks in y 
                (default = grid\_x)
            \item[x\_min] \hfill \\ int, Lower limit of the detection 
                frame on the image
            \item[x\_max] \hfill \\ int, Upper limit of the detection 
                frame on the image
            \item[y\_min] \hfill \\ int, Left limit of the detection frame 
                on the image
            \item[y\_max] \hfill \\ int, Right limit of the detection 
                frame on the image
            \item[print\_block\_color] \hfill \\ book, Print the mean 
                color of the block on the raw image
            \item[get\_block\_color] \hfill \\ bool, Enable the 
                get\_block\_color on start
            \item[get\_color\_x] \hfill \\ int,  Initial x position of 
                get\_block\_color block
            \item[get\_color\_y] \hfill \\ int,  Initial y position of 
                get\_block\_color block
            \item[saturation] \hfill \\ int,  Modify saturation of the 
                image
            \item[brightness] \hfill \\ int,  Modify brightness of the 
                image
            \item[colors] \hfill \\ int,int,int|int,int,int|..., given 
                colors to detect in RGB form
            \item[ground] \hfill \\ int,int,int, color of the ground 
                (default = 155,155,155)(white)
            \item[walls] \hfill \\ int,int,int, color of the walls 
                (default = 0,0,0)(black)
            \item[max\_distance\_to\_color] \hfill \\ int, maximal 
                distance to color before it is counted as ground of walls
        \end{description}

    \subsubsection{Keys}

    \begin{description} \itemindent=-15pt
        \item['l'] change minimum (left,up) bounds mode \\
            \begin{tabular}{ll}
                {\bf left } & y\_min -= 2 \\
                {\bf right} & y\_min += 2 \\
                {\bf up   } & y\_min -= 2 \\
                {\bf down } & y\_min += 2 \\
                {\bf 'p'  } & print x\_min,x\_max,y\_min,y\_max
            \end{tabular}
        \item['u'] change maximum (right,down) bounds mode \\
            \begin{tabular}{ll}
                {\bf left } & y\_max -= 2 \\
                {\bf right} & y\_max += 2 \\
                {\bf up   } & x\_max -= 2 \\
                {\bf down } & x\_max += 2 \\
                {\bf 'p'  } & print x\_min,x\_max,y\_min,y\_max 
            \end{tabular}
        \item['b']  change brightness mode \\
            \begin{tabular}{ll} 
                {\bf left } & -5 to brightness \\
                {\bf down } & -5 to brightness \\
                {\bf right} & +5 to brightness \\
                {\bf up   } & +5 to brightness \\
                {\bf 'p'  } & print brightness value 
            \end{tabular}
        \item['s'] change saturation mode \\
            \begin{tabular}{ll}
                {\bf left } & -2 to saturation \\
                {\bf down } & -2 to saturation \\
                {\bf right} & +2 to saturation \\
                {\bf up   } & +2 to saturation \\
                {\bf 'p'  } & print saturation value \\
            \end{tabular}
        \item['c'] get\_block\_color mode \\
            \begin{tabular}{ll} 
                {\bf left } & move get\_block\_color block to the left  \\
                {\bf right} & move get\_block\_color block to the right \\
                {\bf down } & move get\_block\_color block down \\
                {\bf up   } & move get\_block\_color block up \\
                {\bf 'p'  } & print get\_block\_color block position and 
                              comparisons \\
                            & (distance) with mean color of 
                              the block and \\
                            & the given colors
            \end{tabular}
        \item['e'] exit current mode
    \end{description}
