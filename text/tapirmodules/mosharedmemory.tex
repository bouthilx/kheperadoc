\section{Multiple object shared memory}
\label{sec:mosm}

\subsection{Introduction}
\label{sec:mosm:intro}

The Tapir shared memory output module only share one object, but the 
previously build detectors, colored grid detector and maze detector 
need to share multiple objects at the same time. 

The objects will be saved in a shared memory segment at a given index 
(key). This address (key) can be used in another program to fetch the 
information.

\subsection{Allocating and saving shared memory segment}
\label{sec:mosm:algo}
Here is the algorithm described in chronological order.

    \begin{enumerate}
        \item It allocates a shared memory segment with shmget during 
            initialization.
        \item It assigns an index of the shared memory segment for the list 
            of mutexes. (see section \ref{sec:mosm:ms})
        \item It assigns an index of the shared memory segment for the list 
            of objects. (see section \ref{sec:mosm:ms})
        \item During send (set value in memory), for each $\mbox{object}_i$ in objects:
        \begin{enumerate}
            \item It locks the $\mbox{mutex}_i$ of the $\mbox{object}_i$.
            \item Saves the $\mbox{object}_i$.
            \item Releases the $\mbox{mutex}_i$.
        \end{enumerate}
    \end{enumerate}

\subsection{Memory structure}
\label{sec:mosm:ms}

On the machine I used, sizeof(mutex) was 40 bytes and sizeof(double) 8 
bytes. I could have used only one mutex for all objects, but it would 
have slowed down the process. In overall, I never needed to share more 
than 100 objects. For every objects, there is 5 double, for the slower 
version it would take:
\begin{displaymath}
                100 * 5 * 8Byte +40Byte \simeq 4000 Byte = 4 KB 
\end{displaymath}

For the faster version, we add a mutex for every object, thus it takes:
\begin{displaymath}
                        100 * (5*8Byte +40Byte) = 8000Byte = 8KB
\end{displaymath}

It takes 2 times more space, but 8KB is still really small, so it is 
not a problem. Faster is better.
\\
\\
\begin{displaymath}
    n = \mbox{number of objects}
\end{displaymath}
\begin{center}
\begin{tabular}{c|c|c}
\label{tab:mosm:ms}
   {\bf Indices }   & {\bf Items }  & {\bf Attributes} \\ \hline
   0                & $\mbox{mutex}_1$  & mutex \\ \hline
   $\vdots$         & $\vdots$          & $\vdots$      \\ \hline  
   $n$              & $\mbox{mutex}_n$  & mutex \\ \hline
   $n+1$            & $\mbox{object}_1$ & x \\ \hline
   $n+2$            & $\mbox{object}_1$ & y \\ \hline
   $n+3$            & $\mbox{object}_1$ & size \\ \hline 
   $n+4$            & $\mbox{object}_1$ & angle \\ \hline
   $n+5$            & $\mbox{object}_1$ & runtime \\ \hline
   $\vdots$         & $\vdots$ & $\vdots$ \\ \hline
   $n + (n-1)*5 +1 = (n-1)*6+2 $ & $\mbox{object}_n$ & x \\ \hline
   $n + (n-1)*5 +2 = (n-1)*6+3$ & $\mbox{object}_n$ & y \\ \hline
   $n + (n-1)*5 +3 = (n-1)*6+4$ & $\mbox{object}_n$ & size \\ \hline 
   $n + (n-1)*5 +4 = (n-1)*6+5$ & $\mbox{object}_n$ & angle \\ \hline
   $n + (n-1)*5 +5 = 6*n$ & $\mbox{object}_n$ & runtime
\end{tabular}
\end{center}

\subsection{How to use}
\label{sec:mosm:howto}

Define the number of object that needs to be send and write it in the 
configuration file. Define a unique key\_id and key\_path combination 
that will be used in the tapir detector in CLsquare.

\subsubsection{Parameters config file}
\label{sec:mosm:howto:params}

    \begin{description} \itemindent=-15pt
        \item[key\_path] pathname for the key of shared memory segment 
                (read about ftok())
        \item[key\_id] key for the shared memory segment (read about ftok())
        \item[nb\_objects] the number of objects to be send
    \end{description}
